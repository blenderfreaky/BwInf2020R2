\documentclass{article}

\usepackage{mathtools}
\usepackage{amsmath}
\usepackage{amssymb}

\newcommand{\one}{\frac{d}{d}}

\begin{document}
    
\section{Lösungsidee}

Die Lösung basiert auf der Idee, alle Terme, die aus \(n\) Ziffern bestehen zu errechnen, basierend auf Kombinationen von zwei Termen mit \(i\) und \(j\), \(i+j=n\) Ziffern.
Hierbei muss für jede rationale Zahl \(\frac{a}{b}\in\mathbb{Q}\), \(a,b\in\mathbb{N}\) maximal jeder Term mit \(2(a+b)\) Ziffern berechnet werden, also terminiert der Algorithmus immer.
Dies gilt, da für jede Ziffer \(d\) gilt:
\[\frac{a}{b} = \frac{\one + \one + \ldots + \one}{\one + \one + \ldots + \one}\]

\end{document}