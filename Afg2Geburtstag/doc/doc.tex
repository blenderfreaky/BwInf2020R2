\documentclass{article}

\usepackage[utf8]{inputenc}
\usepackage[german]{babel}

\usepackage{mathtools}
\usepackage{amsmath}
\usepackage{amssymb}
\usepackage{ntheorem}

\newtheorem{theorem}{Theorem}[section]
\newtheorem{corollary}{Corollary}[theorem]
\newtheorem{lemma}[theorem]{Lemma}

\theoremstyle{nonumberplain}
\theoremheaderfont{\itshape}
\theorembodyfont{\normalfont}
\theoremseparator{.\,—}
\theoremsymbol{}
\newtheorem{proof-wo}{Beweis}
\theoremsymbol{\ensuremath{\color{lightgray}\blacksquare}}
\newtheorem{proof}{Beweis}

\begin{document}
    
\section{Lösungsidee}

Die Kernidee des Algorithmus ist es, alle Terme die mit \(n\) Ziffern optimal ihren Wert darstellen zu ermitteln.
Dies wird induktiv gelöst, wobei für die Errechnung der optimalen Terme mit Größe \(n\) alle kleineren Terme vorausgesetzt werden.
Zuerst werden alle infrage kommenden Terme mit Größe \(n\) gebildet, indem alle Terme von Größe \(i\) mit allen Termen von Größe \(n-i\) mit allen binären Operationen gekreuzt werden.
Daraufhin werden alle Terme entfernt, deren Wert schon mit kleineren oder gleich großen Termen dargestellt wurde.
Um mit diesem Verfahren nun die Aufgabe zu lösen ermittelt man solange größere Terme, bis der gesuchte Wert optimal von einem Term dargestellt wird.

Der Algorithmus scheint auf ersten Blick eine stark explosive Komplexität zu besitzen, jedoch ist dem tatsächlich garnicht so.
Im folgenden wird die Komplexität dieses Verfahrens genauer betrachtet.

\newcommand{\termsetn}[1]{\mathbb{T}_{d,b,{#1}}}
\newcommand{\termset}{\termsetn{n}}

\newcommand{\fad}{\forall d, b:\ }
\newcommand{\measure}[1]{\varphi_{d,b}\left(#1\right)}

\subsection{Term-Menge}

Die Term-Menge \(\termset\) enthält alle Terme die \(n\) mal die Ziffer \(d\) in der Basis \(b\) (meistens \(= 10\) für Dezimalsystem) verwenden.
Die Elemente der Term-Menge \(\termset\) sind 3-Tupel bestehend aus einer Operation und zwei Termen der Größe \(i\) und \(n-i\) oder \(n\)-fache Wiederholungen der Ziffer \(d\).

\[\termset :=
\bigcup_{i=1}^{n-1} \left\{ (l, r, o) | l \in \termsetn{i}, r \in \termsetn{n-i}, o \in \{ +, -, \cdot, \div \} \right\}
\cup \left\{ \left(\sum_{i=0}^n b^n \cdot d \right) \right\} \]
Weiterhin ist der Spezialfall \(n=0\) als die leere Menge definiert:

\[\termsetn{0} := \{\}\]

\subsection{Evaluationsfunktion}

Basierend auf der Term-Menge ist die Evaluationsfunktion \(\epsilon : \termset \to \mathbb{Q}\) definiert, die den Wert eines Terms ermittelt:
Für rationale Zahlen \(x\in\mathbb{Q}\) gilt:
\[\epsilon(x) = x\]
Ansonsten gilt:
\[\epsilon((l, r, o)) = o(l,r)\]

\subsection{Größenmaß-Funktion}

Zur Berechnung der Komplexität wird die Größenmaß-Funktion \(\varphi_{d,b} : \mathbb{Q} \to \mathbb{N}\) betrachtet, die die Anzahl an Verwendungen der Ziffer \(d\) errechnet, die für die optimale Repräsentation beliebiger rationaler Zahlen benötigt wird.
Genauer formuliert ist \(\measure{x} := \min_{i\in\mathbb{N} \land x \in \termsetn{i}} i \)
Beispielsweise ist \(\fad\measure{d} = 1\).

\begin{corollary}
    Aus der Definition lassen sich direkt folgende
\end{corollary}

\begin{lemma}[Triviale obere Grenze des Größenmaß]
Für jede rationale Zahl

\[s\cdot\frac{a}{b}\in\mathbb{Q},\ a,b\in\mathbb{N}\backslash\{0\},\ s\in\{-1,0,1\}\]
gilt:

\[\fad \measure{s\cdot\frac{a}{b}} \leq a + b + 2\]
\end{lemma}

\begin{proof-wo}
Für jede Ziffer \(d\) gilt:

\[\frac{a}{b}
= \frac{1 + 1 + \ldots + 1}{1 + 1 + \ldots + 1}
= \frac{d\cdot(1 + 1 + \ldots + 1)}{d\cdot(1 + 1 + \ldots + 1)}
= \frac{\overbrace{d + d + \ldots + d}^{a}}{\underbrace{d + d + \ldots + d}_{b}}\]
\[\implies \measure{\frac{a}{b}} \leq a + b\]

weiterhin gilt auch:

\[s\cdot\frac{a}{b} = 
\begin{cases}
    \frac{d + d + \ldots + d}{d + d + \ldots + d} & s = +1 \vspace{0.2em}\\
    d-d & s = 0 \vspace{0.2em}\\
    (d-d) - \frac{d + d + \ldots + d}{d + d + \ldots + d} & s = -1 \\
\end{cases}\]

Somit lässt sich jede positive rationale Zahl mit \(a+b\) oder weniger und jede negative rationale Zahl mit \(a+b+2\) oder weniger Ziffern darstellen. Null lässt sich immer mit 2 Ziffern darstellen.
Folglich lässt sich jede rationale Zahl \(s\cdot\frac{a}{b}\) mit \(a+b+2\) oder weniger Ziffern darstellen.
\end{proof-wo}



\end{document}